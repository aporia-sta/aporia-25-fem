\vspace*{\credgap}
{\noindent\LARGE\sc Letter from the Editor}
\vspace{\credgap}

\vspace{\ackgap}\noindent
Dear reader,

\vspace{\credgap}\noindent
As my work on this year’s Feminist Edition comes to an end, it is
tempting to speak of it only fondly. After all, as a philosopher, it
has been an intensely joyful experience to work with such a talented
team of editors, writers and philosophers to produce this year’s
edition.

However, as a feminist academic, this year has been challenging.
Misogyny and gendered violence against women is on the rise, and
academic work studying women’s health, lives and experiences is being
defunded, censored and framed as ‘frivolous’ or ‘unnecessary’. It is
not. This work – our work as feminist philosophers - undergirds
general feminist efforts in an essential way. It grants us the
theoretical tools to grapple with viscerally real problems, and it
therefore is in times like these that our research matters most.

As such, I would like to extend my immense gratitude to all the
philosophers who submitted their research and writing to The Feminist
Edition this year, as well as my congratulations to those who were
published. Your inquiry matters, and I am very proud to have worked to
provide a platform where your thoughts can be shared.

I also deeply thank this year’s editorial team, with a special tip of
the hat to my highly skilled colleagues Kirsty Graham and Joe
Bradstreet. You have all volunteered your time and brains over the
course of this year towards the noble cause of platforming young
academics and helping them develop their philosophical acumen. Thank
you so much for your hard work – it makes a difference.

Lastly, like last year, I want to express my gratitude to – and awe
for - all the inspiring women and feminist academics in the Philosophy
Society as well as the St Andrews Philosophy Department. To be
surrounded by you all is all the affirmation needed to know that what
we are doing here is worthwhile, and an academic necessity.

If you made it this far, I must also say – it is with great sadness I
now leave The Feminist Edition behind, but with great joy I leave it
in very competent hands. I cannot wait to see what the future holds
for this journal and the people who work so hard to bring it forth.

\vspace{\credgap}\noindent
Signing off,

\vspace{\ackgap}\noindent
Christina Landys Herre

\noindent
Editor-in-Chief, \emph{Aporia: The Feminist Edition}

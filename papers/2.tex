\chapter{Criminalizing `Unjust Sex'}
\chaptermark{Criminalizing 'Unjust Sex'}
\chapterauthor{Sofia Mona,
  \textit{University of St. Andrews}}

\titleformat{\section}[block]{\normalfont\scshape\large\bfseries}{}{0pt}{}
\begin{quote}
    
This essay examines the limitations of current rape law and advocates
for legal reform to better protect sexual autonomy. Sexual autonomy,
defined as the right to freely choose and refuse sexual interactions, is
foundational to liberal legal principles. However, the concept of
`unjust sex', which involves manipulation, coercion, or exploitation of
agency without physical force, reveals gaps in existing legal
frameworks. Drawing on Ann J. Cahill's work, the essay argues that
unjust sex undermines agency and autonomy, causing significant harm that
warrants criminalization. While rape nullifies sexual autonomy outright,
unjust sex limits the individual's capacity for meaningful
self-determination, reinforcing systemic power imbalances. The essay
addresses concerns about potential overreach, arguing that criminalizing
unjust sex defends autonomy without imposing moralistic control. It
concludes that protecting sexual autonomy requires acknowledging and
addressing the harm caused by unjust sex.
\end{quote}

\section*{Introduction}
Sexual autonomy -- the ability to choose and shape the sexual relations
one has -- is a right as fundamental as any other type of autonomy and
is legally protected. The law on sexual offences defines them as
`violations of the right to sexual self-determination' (Hörnle 2016,
851). Following a liberal perspective, it has been set out to strengthen
sexual autonomy by loosening the grip of the law and decriminalizing
certain sexual acts, e.g. homosexuality and adultery. The notion of a
liberal criminal law concerning sexual offences has thus been strongly
associated with decriminalization, especially in the second half of the
20\textsuperscript{th} century. In this essay, however, I will argue for
the reform of laws pertaining to sexual offences to better protect
sexual autonomy by criminalizing sexual acts lacking valid and robust
consent including forms of `unjust sex', as termed by Ann J. Cahill
(2016). The argument will follow from the description of unjust sex as
undermining the victim's agency, which is crucial for the establishment
of autonomy. I will argue that it can be in the interest of a liberal
theory of law to criminalize more, in order to protect the legal asset
of sexual autonomy. First, I will introduce the notion of sexual
autonomy and its dual dimensions of positive and negative liberty. Next,
I will provide a brief historical overview of how the focus of rape law
has evolved over time, highlighting the shift from a focus on marital
rights to a recognition of autonomy and consent as central concerns.
Then, I will discuss the concept of unjust sex and why it undermines
sexual agency. I will argue that unjust sex represents a significant
harm to sexual autonomy which justifies its criminalization. Finally, I
will address concerns about the criminalization of unjust sex and
conclude.

\section*{Sexual Autonomy}
Humans have a right to sexual autonomy, as much as they have a right to
autonomy in general. The desire to be able to choose and control the way
in which one engages in sexual activities is a core characteristic of
human sexuality. It allows the individual to express themselves in a
certain way while also allowing other people to take part in a most
intimate area of the human body and psyche. This right, however, must
always be understood in relational terms that involve all sexual
partners. Everyone concerned has the right not to have their right to
sexual autonomy overridden. Thus, sexual autonomy is restricted by the
sexual autonomy of others (Schulhofer 2000, 99).

Sexual autonomy is discussed in terms of two notions of liberty or
freedom. Firstly, it includes the positive liberty, i.e. the freedom, to
engage in consensual acts according to one's own desires and needs
(Hörnle 2016, 859). This is an important aspect of liberal thought, the
idea that it is the individuals themselves who can shape their
expression of sexuality how they wish. It is crucial, however, that
consent is established. Otherwise, the autonomy of the sexual partners
involved is compromised. This is where negative freedom enters: Negative
freedom when it comes to sexual interactions is the right to refuse
participation in sexual acts at any time. It is the right not to be
exposed to the actions of other people that one does not want to
participate in or be subjected to (ibid). It is also the right of
defence -- if someone coerces you into engaging in a sexual act that you
do not want, you are allowed to defend yourself. The negative freedom to
sexual autonomy also imposes on others a duty not to interfere; it
limits their positive liberties. It is only through consent that the
duty not to interfere can be removed, it is consent that makes
interferences, i.e. the sexual interactions, permissible and legal
(Scheidegger 2021, 771).

The right to sexual self-determination, especially for women, must also
be understood in a historical context. Originally, the criminal law on
sexual offences was established to safeguard the authority of fathers
and husbands over women\textquotesingle s bodies. Women were their
property, and in cases of rape, it was possible that the woman would get
accused of adultery and thus harm the honour of her husband (Lameyre
2000, 92--93). Rape within a marriage was largely inconceivable as the
husband had full sexual rights over his wife. To free herself from this
accusation and to free her father or husband from dishonour, the woman
had to prove an element of coercion, which demonstrated that she had
shown sufficient resistance to the aggressor. Even though this notion is
highly outdated now, the requirement of coercion as a central element of
rape law is now being removed in many jurisdictions, albeit only after a
prolonged struggle and resistance from a patriarchal society
(Scheidegger 2021, 770). But there is no denying that there has been a
significant change in the attitudes towards rape law almost on a global
scale in the last twenty or thirty years. Indeed, the focus of rape law
has shifted from coercion as the main characteristic of rape to a
consent-based model, that defines rape as sex against one's
will.\footnote{This is the case in most European countries (see, e.g.,
  `Europe: Spain to Become Tenth Country in Europe to Define Rape as Sex
  without Consent' 2020)} This already includes the idea of sexual
autonomy; the right to choose the sex you want. Thus, as Tatjana Hörnle
puts it, disregarding the right to sexual autonomy is punishable as such
(Hörnle 2016, 862). What is punishable is the offence against sexual
autonomy. Rape law has thus moved away from a moralizing perspective
that dictated with who and how sex was permissible or not, to a focus on
sexual autonomy, emphasizing the individual's right to shape their own
sexual interactions.

Following the notion of positive freedom in relation to sexual offences,
there has been a clear tendency to decriminalize certain sexual acts.
For example, the abolition of criminal offences of adultery, sodomy,
homosexuality and incest (Scheidegger 2021, 770). One main idea is that
the state has no right to determine or have control over the way the
individual wants to have sex. This has also led to a demoralization and
destigmatisation of certain sexual relations. Thus, positive freedom has
the effect - at least in tendency - that we criminalize less. Being able
to act according to your own wishes and needs primarily means that the
state should not interfere in this intimate area and should be tolerant
and non-paternalistic. Autonomy in the sense of positive freedom is the
epitome of a modern law on sexual offences: de-moralization of sexual
criminal law, getting away from religious commandments and hence
decriminalization. This is also reflected in how the law is named: What
in some legal orders was termed ``offences against morality'' became
``offences against sexual autonomy'' (Hörnle 2016, 851). Since today, it
is sex against the will or sex without consent that is punishable,
which, at its core, embodies the idea that sexual autonomy is worthy of
protection and should be able to shape sexual interactions in a
meaningful way, it seems as if everything is settled. However, there are
still cases of impairment of sexual autonomy that are not clearly
covered by the reformed law. One complex set of these cases can be
collected under the heading of `unjust sex'.

\section*{Unjust Sex}
Ann J. Cahill discusses the differences between `unjust sex' (as termed
by Nicola Gavey, 2005) and rape and introduces the idea that the
victim's agency is crucial in determining whether an act is rape or
not.\footnote{Cahill limits her discussion on hegemonic heterosex, and
  the following descriptions are of a very gendered nature that
  stereotypically portray heterosexual cis-women as the victims and
  heterosexual cis-men as the offenders (Cahill 2016, 747).} There are
some heterosexual interactions that occupy a `gray area', where sex
occurs under pressure (albeit non-violently) or with passive
acquiescence, making them ethically problematic but distinct from rape.
Examples include ``situations in which a man applied pressure that fell
short of actual or threatened physical force, but which the woman felt
unable to resist'' (Gavey 2005, 136). The elements of ``letting sex
happen'', or ``going along with sex'' shape the interactions of unjust
sex that fall into this `gray area' (ibid). These sexual interactions,
though not overtly violent, are nonetheless not desired and thus
possibly non-consensual. What permeates the descriptions is the notion
of giving in and conceding to the actions. The sexual interactions are
accompanied by a sense of moral wrongness which cannot simply be equated
with rape. While there are common elements between sexual assault and
certain forms of unjust sex, such as coerced and pressured sex, they
differ significantly in terms of the role and efficacy of the victim's
sexual agency. In instances of unjust sex, the victim's sexual agency is
acknowledged but constrained or exploited, serving as a superficial
validation of the interaction. Feeling like there is no way of refusing
sex and `having to go along with it' demonstrates a constraint on the
ability to act freely. In sexual assault, agency is overridden or
nullified (Cahill 2016, 758). The victim's right to not engage in the
sexual interaction or the right not to be interfered with is revoked and
constitutes a harm to sexual autonomy. It is important to acknowledge
that women have sexual agency and that denying them this agency
constitutes serious harm.

What Cahill highlights is the fact that what makes rape problematic is
the nullification of the victim's sexual agency. Since sexual autonomy
and agency are intrinsically connected, a harm to agency is also a harm
to autonomy (Cahill 2016, 757; 2016, 760). Autonomy relies on the
ability to make meaningful choices, and when agency is constrained, the
capacity for autonomous decision-making is diminished or abolished. The
nullification of the victim's sexual agency can be seen as giving enough
grounds for criminalization as it amounts to reprehensible sex against
the will, to sex that prevents the individual from acting autonomously,
i.e. to non-consensual acts. As undermining sexual agency is harmful in
at least this very important sense, unjust sex should be criminalized as
it is contrary to sexual autonomy.

Cahill also points out, that sexual agency is to be understood in
relation to others (Cahill 2016, 757). This means that agency is not
exercised in isolation but is shaped by and interacts with the agency of
others. The agency is limited by the duty of non-interference imposed by
the positive right of others to sexual autonomy. This is similar to how
autonomy is described, and that the relational aspect of sexual
interactions is limited by the autonomy of others.

\subsection*{False affirmations of autonomy}
There are cases where autonomy can be weaponized: in unjust sex, the
appearance of autonomy -- where a woman's consent or acquiescence is
sought -- can paradoxically undermine her autonomy. This occurs when her
`choice' is used to validate an interaction that does not genuinely
respect or expand her sexual agency. In such instances, consent becomes
a tool for masking manipulation or coercion rather than an expression of
free will. It is fictitious and non-valid consent, but one that is
difficult to detect because it hides behind the façade of proper
consent. There can be instances of manipulation or non-violent coercion
that lead to this kind of ostensible consent. It can also be the case
that preexisting power dynamics influence the `choice making' but in a
way that does not further the victim's sexual agency. When a person's
consent is shaped by factors like economic dependence, social pressure
or emotional vulnerability, the resulting action may appear consensual
but fail to respect or affirm their autonomy. Rae Langton describes how
affirming someone's autonomy when it is actually constrained can mask
the underlying coercion and power imbalance (2009, 14). This
recognition of false or apparent autonomy reinforces systemic
injustices.

Indeed, traditional gender roles and expectations often normalize
behaviours that subtly undermine sexual agency framing them as
acceptable aspects of romantic love. Consider the idea of the male
pursuer, having to woo the woman and interpret her refusal or denial as
teasing him, not understanding that she is setting boundaries and
asserting sexual autonomy. Or the expectation that women have to
prioritize male pleasure and give in to pleas in order to avoid
conflict; these norms all lead to a normalization of violation of sexual
agency. Without recognizing and addressing the limitations on sexual
agency in everyday sexual interactions, we cannot adequately confront
the culture that allows the undermining of sexual agency and autonomy to
perpetuate.

\section*{Synthesis}
Combining the notions of both sexual autonomy and unjust sex, this is
what results: the legal asset which criminal law on sexual offences aims
to protect is sexual autonomy. Sexual autonomy is the right to engage in
consensual sexual interactions that one desires without interfering with
the negative freedom of the sexual partners not to be coerced into acts
that they do not want to participate in. In instances of rape, sexual
autonomy is nullified, taking away the right to sexual autonomy of the
victim. The violation is complete and leaves no room for the exercise of
autonomy. By contrast, in cases of unjust sex, agency is acknowledged
but deliberately exploited, limiting the victim in their ability to
assert their sexual autonomy. Although this may not nullify autonomy in
the same way as rape, it imposes significant restrictions on the
victim's capacity for self-determination, thereby causing harm to their
sexual autonomy. The harm caused by unjust sex is not merely moral but
involves a legal and social dimension that requires recognition and
intervention. Since sexual autonomy is what the law aims to protect, I
conclude that there are compelling reasons to criminalize unjust sex.

\section*{Concerns}
This line of argument raises several significant concerns, which I will
address in this section.

Firstly, the tension between the liberal idea of decriminalization and
the lived reality of many women still experiencing cases of `unjust sex'
that are not captured by existing rape law can lead to confusion.
Stricter penalties for offences can provoke defensive reflexes in
liberal-minded people. The fear is that increased criminalization might
lead to overreach or moral paternalism. The deeply intimate nature of
sexuality often leads to an intuitive resistance against state
interference, as it is seen as an area where the state should not
interfere, and the suspicion of moralization is high. Excessive state
control over private lives raises issues about how much state
interference is allowed and can be tolerable when it comes to regulating
intimate relationships.

This concern is understandable, particularly given the historical
trajectory of law on sexual offences. Liberal thought has long
emphasized decriminalization as a means of protecting individual
freedoms, ensuring that the state does not impose moral judgments on
private sexual behaviour. However, the expansion of the law to include
unjust sex is not a step towards moralizing sexual interactions but a
necessary measure to uphold sexual autonomy. The criminalization of
unjust sex does not aim to regulate private morality but to safeguard
individuals' ability to make autonomous choices in sexual interactions.
The crucial distinction lies in the law's objective: it is not concerned
with evaluating the moral worth of particular sexual acts but with
preventing coercive and exploitative behaviour that undermines sexual
authority. Unlike past laws that sought to impose moral norms -- such as
those criminalizing homosexuality or adultery -- the proposed reform is
rooted in the principle that individuals should be able to make free
choices about their sexual interactions. Furthermore, the argument for
criminalizing unjust sex does not contradict the liberal commitment to
limiting state interference in private life. On the contrary, it aligns
with it. Liberalism is fundamentally concerned with ensuring that
individuals can exercise autonomy without coercion or undue influence.
The same rationale that justifies criminalizing rape -- protecting
individuals from violations of their autonomy -- also supports
addressing unjust sex, as both involve the imposition of unwanted sexual
interactions. By criminalizing unjust sex, the law does not overreach
but instead ensures that sexual autonomy is meaningfully protected,
reinforcing the very principle of individual's self-determination that
liberalism upholds. A progressive, liberal law on sexual offences does
not necessarily rely on decriminalization.

A further concern is the potential blurring of boundaries between unjust
sex and rape. Nora Scheidegger highlights the importance to reserve the
term ``rape'' for the most egregious violations of sexual autonomy
(Scheidegger 2021, 783). Conflating the two could dilute the power and
significance of the term `rape'. Rape is considered one of the most
reprehensible violations of sexual autonomy and widening the scope of
its definition might weaken its impact. This is similar to the way
psychiatric terms like `depression' are sometimes used casually even
when they don't apply, which diminishes their power and seriousness. The
concern is that labelling too many behaviours as rape might trivialize
its profound harm and undermine public and legal recognition of its
severity.

In response to this objection, one could reply that the argument was not
aimed at putting unjust sex into the same category as rape. I have
argued that unjust sex as such should be criminalized, and not that
because unjust sex equals rape, it should be criminalized. As a distinct
category of sexual offences, unjust sex harms sexual autonomy and should
be criminalized on the grounds of exactly this. Thus, it is more
effective to create distinct legal categories to address non-violent
abuses effectively.

Tied to the idea of creating new legal categories is the obvious concern
of how the case of unjust sex can be proved in a legal setting. Unjust
sex occurs in a more ambiguous space where coercion is subtle, and
consent may appear to be given, even if it is influenced by pressure or
manipulation. This raises significant questions on how to go about
evidence and proof, which are already challenging in sexual offence
cases due to their reliance on conflicting testimonies -- often leading
to legal deadlock in `he said, she said' scenarios. How can the law
reliably distinguish between an individual who truly consents and one
who `goes along with' sex?

A potential response would be to carefully define unjust sex in legal
terms, ensuring that it is distinguished both from consensual sex and
legally recognized forms of sexual assault. This would involve
distinguishing criteria for recognizing coercion beyond physical force,
such as establishing a threshold for undue pressure, manipulation or
abuse of power. A clear formulation of how the criminalization of unjust
sex should be approached, however, is an undertaking that pushes the
limits of this essay.

\section*{Conclusion}
In conclusion, I have argued that unjust sex -- instances of sexual
interactions that are characterized by exploitation of sexual agency --
should be criminalized on the grounds that they harm sexual autonomy.
Sexual autonomy, as a legal and moral principle, underpins the ability
to make meaningful and voluntary choices when it comes to sexual
interactions. It is characterized by the positive freedom to choose
which sexual interactions to engage in and the negative freedom to
refuse to participate in sexual interactions. Unjust sex, by
manipulating or undermining agency, violates this principle, reducing
the individual's capacity for self-determination. Cases that allegedly
fall into the `gray area' have to be painted in colour and acknowledged
as acts that undermine sexual agency in a way that threatens and harms
sexual autonomy. Not only do these instances reinforce behaviour that
perpetuates systemic power imbalances, but they also contribute to a
culture that normalizes the undermining of sexual agency. The
criminalization of unjust sex is not an overreach but rather a necessary
measure to make sure that sexual autonomy is protected. The effort is
aimed at defending autonomy and not at imposing moralistic control. An
obvious question arises from this analysis: how do we go about
criminalization? This, however, is a topic best reserved for a separate
discussion.

% restore title format like nothing ever happened
\let\section\sectionDefault
\let\titleformat\titleformatDefault

\refsection

\begin{hangparas}{\hangingindent}{1}
Cahill, Ann J. 2016. `Unjust Sex vs. Rape'. \emph{Hypatia} 31 (4):
746--61.

`Europe: Spain to Become Tenth Country in Europe to Define Rape as Sex
without Consent'. 2020. 3 March 2020.
https://www.amnesty.org/en/latest/news/2020/03/europe-spain-yes-means-yes/.

Gavey, Nicola. 2005. \emph{Just Sex? The Cultural Scaffolding of Rape}.
1. publ. Women and Psychology. London: Routledge.

Hörnle, Tatjana. 2016. `Sexuelle Selbstbestimmung: Bedeutung,
Voraussetzungen Und Kriminalpolitische Forderungen'. \emph{Zeitschrift
Für Die Gesamte Strafrechtswissenschaft} 127 (4). \newline
https://doi.org/10.1515/zstw-2015-0040.

Lameyre, Xavier. 2000. \emph{La criminalité sexuelle}. Dominos 206.
Paris: Flammarion.

Langton, Rae. 2009. \emph{Sexual Solipsism: Philosophical Essays on
Pornography and Objectification}. Oxford\,; New York: Oxford University
Press.

Scheidegger, Nora. 2021. `Balancing Sexual Autonomy, Responsibility, and
the Right to Privacy: Principles for Criminalizing Sex by Deception'.
\emph{German Law Journal} 22 (5): 769--83. \newline
https://doi.org/10.1017/glj.2021.41.

Schulhofer, Stephen J. 2000. \emph{Unwanted Sex: The Culture of
Intimidation and the Failure of the Law}. Cambridge, Massachusetts
London: Harvard University Press.
\end{hangparas}

%%% Local Variables:
%%% mode: LaTeX
%%% TeX-master: t
%%% TeX-master: "../main"
%%% End:

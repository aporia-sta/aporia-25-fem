
\chapter{Why Women and Men Cannot Love Each Other (Yet)}
\chaptermark{Why Women and Men Cannot Love Each Other}
\chapterauthor{Audrey Rodriguez,
\textit{University of Miami}}

% makes the section numbers roman numerals
\renewcommand{\thesection}{\Roman{section}}

% makes the subsection letters
\renewcommand{\thesubsection}{\alph{subsection}.}

\begin{quote}
In a heteronormative society, men and women are
typically expected to look not for authentic love, but simply a partner
of the opposite gender. This compulsory heterosexuality, as explained by
Adrienne Rich, and the resultantly tainted love story problematize views
about love like Berit Brogaard's ``appraisal respect''. I take Brogaard
to give an apt account of what we should want authentic love to be, one
in which we are said to love another when we properly evaluate their
role as a lovable lover. However, because loving another and evaluating
their lovability are not the goals of love as it stands, heterosexual
men and women cannot be said to love in the way Brogaard rightly
champions. Authentic love is then something most do not generally
experience, but all (who are interested in engaging in romantic love)
ought to strive for. I ultimately claim that developing respect for
ourselves, our peers, our same-sex relationships, and love itself are
the best ways for us to make authentic love widely accessible.
\end{quote}

\vspace{\credgap}

\noindent In a heteronormative society, men and women are
typically expected to look not for authentic love, but simply a partner
of the opposite sex. Can you be said to love your partner without truly
getting to \emph{choose}\footnote{My argument throughout this work pressupposes at least a minimal amount of free will. What authentic love would look like in a hard determinist picture is an interesting question, but whose answer is opaque enough that I will not be endeavoring to answer it here.} your partner? Many feminist theorists
have taken issue with whether men can love women under patriarchy since
patriarchy does not see women as ends-in-themselves, but the reverse
case has rarely been considered.

I argue that women are also not taught to strive to love men, but taught
to objectify men as a means to the securing of connection to a
subjectivity. Heterosexual love is thus an inauthentic experience for
heterosexual men and women alike. This is because heterosexual love
projects, as they stand, necessarily hold not love as their purpose; but
rather the fulfillment of societal expectations.

In Section I of this paper, I will explain the constraints compulsory
heterosexuality places on love. In Section II, I will recount Berit
Brogaard's framework describing romantic love as a goal-oriented emotion
that is importantly different from friendship\footnote{Throughout this
  paper I will refer to ``platonic love'' as ``friendship love'' in
  keeping with the terminological choice of one of the main authors with
  whose work I am interacting, namely, Berit Brogaard (2022). Any
  instance of ``friendship love'' can be understood to refer to the same
  love between friends that the phrase ``platonic love'' picks out.}
love. I will use the problem of compulsory heterosexuality to complicate
Brogaard's assumption that the appraisal of one's performance in the
role of lover accounts for lovers' ability to respect each other when
engaging in romance is generally possible.

It will become clear that most do not yet have the type of respect
necessary to be said to love authentically, and in Section III I will
argue that men and women cannot generally love each other in an
authentic sense. I will use the phrases ``genuine love''\footnote{Bauer,
  Nancy. \emph{Simone de Beauvoir, Philosophy, \& Feminism}. New York:
  Columbia University Press, 2001. 164-165.} and ``authentic
love''\footnote{Bauer,  Nancy. \emph{Simone de Beauvoir, Philosophy, \& Feminism}. New York: Columbia University Press, 2001. 164.} interchangeably to refer to a love that
is genuine/authentic in so far as it ``is an expression of the highest
of moral laws: when I love another person genuinely I both exercise my
existential freedom and evince the highest respect for the freedom of
other, on which, I understand, my own freedom rests.'' (Bauer, 164--5)
This respect for another's freedom is something I take to be most
clearly portrayed by Brogaard's lovability account, and something that
clearly seems to be a necessary aspect of a kind of love worth having.
These oppressive societal constraints also make heterosexual friendship
love generally impossible according to the ``appraisal respect''
standard. Finally in Section IV, I will consider general objections to
my claims, offer responses, and consider ways in which we could
eventually create the conditions for and ultimately secure an authentic
heterosexual love.

\section{Compulsory Heterosexuality}

Adrienne Rich writes in her essay ``Compulsory Heterosexuality'' that
heterosexuality is a ``\emph{political institution}'' that dictates that
women must be attracted to and pursue relationships with men so as to
assure the ``male right of physical, economical, and emotional access''
to women.\footnote{Rich, ``Compulsory Heterosexuality and Lesbian
  Existence.'' 647.} To deny patriarchy's requirement of heterosexual
love from women is often to open oneself up to ``physical torture,
imprisonment, psychosurgery, social ostracism, and extreme
poverty.''\footnote{Rich, ``Compulsory Heterosexuality and Lesbian
  Existence.'' 653.} Heterosexuality is then required of
women not only at threat of discomfort while in the confines of
patriarchy, but at the risk of a woman's mental, social, and physical
safety. All those who live under patriarchy are indoctrinated to believe
the only form of romantic love that is common, ``normal,'' or worthy is
heterosexual in nature.

The coercive power of this expectation of heterosexuality is so strong,
in fact, that it becomes completely compulsory. With the compulsion of
heterosexuality in romantic love, and the definition of romantic love
thus being inextricable from a heterosexual relationship structure, this
means love itself becomes compulsory as does its structure. One cannot
be said to truly be making a choice when only given one option, and one
cannot be said to truly engage in loving when only given one definition
and version of love. Therefore, those in most heterosexual relationships
cannot be said to truly be loving. Instead, many are unwittingly
engaging in a societally mandated project akin to military enlistment.

\subsection{Why Heterosexual Love is In Question}

Heterosexual love is forced in a way most other types of love are not. I
have been asked many times why I take most issue with heterosexual love
if starting from an asymmetry in respect or societal power. There are
many romantic relationships that can span any number of other oppressed,
or not oppressed, lines - be these racial, socioeconomic, in terms of
age, etc. I believe many of these are a non-issue in the face of the
account of an ideally respectful love I sketch in Section III.
Addressing the other types of love that still might be questionable even
in the face of such an authentic love is out of the scope of this paper.
Women\footnote{Throughout this paper I will use the terms ``women'' and
  ``men'', and will take both to mean anyone who identifies as either of
  those two genders at least occasionally. Again, there are many
  identity markers that might call for a more fine-grained and specific
  discussion that considers more than just the issues in love between
  binary genders. It is just the general power imbalance between those
  who identify as men and those who identify as women, and the
  compulsory nature of heterosexuality, that I think makes heterosexual
  love one of the most contentious and confounding forms of romantic
  love.} are understood by most to be pervasively defined in terms of
men and generally oppressed by the objectifying structure of this
relation. In the next two Sections I will try to make clear how such a
societal power imbalance and compulsory heterosexuality clearly
problematize heterosexual love given the world as it is now.

The realization of male sexual power ``by adolescent boys through the
social experience of their sex drive'' is the same realization that
causes ``girls [to] learn that the locus of sexual power is
male.''\footnote{Rich, ``Compulsory Heterosexuality and Lesbian
  Existence.'' 645.} Girls come to know their sexual identities through
boys' realization of theirs, making female sexual desire compulsorily
linked to that of men and pleasing men. In a search for any kind of
negotiating power on the societal stage, women become sexual responders
to male power as opposed to explorers and actors of their own desires.
This is all true if one accepts, as many feminists do, that women are
kept subordinate by oppressive structures by patriarchy at best, or that
women are entirely second-class citizens in how they are respected by
societies at large and at worst. Not only are women taught to define
themselves in terms of their ability to appeal to men's sexual appetite,
but they also come to know themselves as objects.

It is in the packaging of heterosexual love in the ``workplace [\ldots]
where women have learned to accept male violation of our psychological
and physical boundaries as the price of survival; where women have been
educated---no less than by romantic literature or by pornography---to
``perceive ourselves as sexual prey.''\footnote{Rich, ``Compulsory
  Heterosexuality and Lesbian Existence.'' 642.} All cultural and
political channels create and fortify compulsory heterosexuality, making
it a cultural and political pillar itself. This enforced and thusly
reinforced self-perception of women as sexual prey causes women to feel
that danger at the hands of men is imminent and the only remedy is
aligning themselves with men in the hopes of being protected.

Rich asks that all women who assume heterosexuality to be innate or a
choice consider that it is in fact ``something that has to be imposed,
managed, organized, propagandized, and managed by force.''\footnote{Rich, ``Compulsory
  Heterosexuality and Lesbian Existence.'' 648.} Heterosexuality is thus not a choice or preference, but
rather it is a regime backed by threat of death, torture, and social
abandonment.

Love and this sexual power imbalance cause women enveloped by compulsory
heterosexuality to see their identity fulfill ``a secondary role and
[grow] into male identification.''\footnote{Rich, ``Compulsory
  Heterosexuality and Lesbian Existence.'' 642.} Female
subordination is then eroticized and the ``access to women only \emph{on
women's terms}'' becomes something unthinkably frightening to
men.\footnote{Rich, ``Compulsory
  Heterosexuality and Lesbian Existence.'' 643.} It is this identification with men, fear of
societal retaliation, and the eroticization of female subordination that
makes women search for themselves by way of being romantically
associated with a man. A woman's difficulty in separating her sexual
drive from that of men becomes part of the love and sex game, with women
having to become accustomed to relinquishing their power of desire to
men. This results in a clear objective laid out for women in engaging in
romantic projects\footnote{I elect to use the term ``romantic projects''
  instead of ``romantic relationships'' because I do not want to confuse
  relationship projects with romantic ones. It seems the former would
  need to factor in more practical matters (longevity of the
  relationship, living arrangements, etc.) than I have space to
  undertake in this project. I would like to leave the definition of
  what a romantic relationship is and questions regarding polyamory and
  how much ``committed'' ``monogamy'' is indicative of a healthy
  relationship open. I merely mean to argue throughout this paper that
  heterosexual love is misunderstood and inappropriately portrayed on a
  societal scale and has little to no authenticity motivating it.}:
securing a subjectivity to which you can attach yourself. This
objectifies men because they become the kind of object, the kind of
thing, that has the kind of subjectivity needed to live more freely, and
women are taught they can only really find power and identity by growing
into a male's identity since their sexual desires and others are defined
in terms of men's desires. Thus, romantic projects are the clearest way
for women to gain societal power and ``love'' so-construed never figures
into the picture.

\section{Love for Lovability's Sake}

Compulsory heterosexuality will thus be the lens through which we come
to understand love, and Berit Brogaard's definition of love will give a
theory to be considered. It is necessary to give a definition of love
that can bring light to the difficulties in squaring the economically
and socially disadvantaged position in which women find themselves with
the idea of engaging in heterosexual love. Brogaard's characterization
also strikes me as the most concrete explanation of what an ideally
authentic, healthy, and genuine love is; which is also that which should
be strived for if romantic love is to be one works towards.

Brogaard situates love as a socially and personally defined emotion in
which ``evaluations of the perceived, remembered, or imagined objects
elicit the bodily and mental changes characteristic of the specific
emotions.''\footnote{Brogaard, \emph{Friendship Love and Romantic Love.}
  171.} Similar to the way in which a fear of heights renders height
scary to some, this ``perceived-response theory of emotions\ldots [makes
it so that] love renders a person as lovable, or worthy of
love.''\footnote{Brogaard, \emph{Friendship Love and Romantic Love.}
  171.} Her account seeks to establish a clear
definition of love that can distinguish romantic and friendship love
while also avoiding relying on a motivational account as such accounts
can lead to the incorrect assumption that heterosexual men tend to
respect the dignity of women who arouse them.\footnote{Brogaard, \emph{Friendship Love and Romantic Love.} 171.}

Brogaard utilizes Stephen Darwall's concept of ``appraisal respect'' to
illustrate her theory that love is a matter of the appraisal of a person
in terms of their moral perfection generally and in a specific
realm.\footnote{Brogaard, \emph{Friendship Love and Romantic Love}. 172.}
Brogaard's theory of love then draws on this concept but diverges in the
defining of the appraisal inherent in love ``in terms of properties we
value in them.''\footnote{Brogaard, \emph{Friendship Love and Romantic Love}. 172.} Brogaard's use of appraisal respect as
opposed to recognition respect designates respect for one's lovability
as an aspect of their character.\footnote{Darwall, ``Two Kinds of
  Respect.'' 41.} Those features of people which Darwall and thus
Brogaard define as ``constituting character'' are ``those which we think
relevant in appraising them as persons'' and ``those which belong to
them as moral \emph{agents}.''\footnote{Darwall, ``Two Kinds of
  Respect.'' 43.} This focus on the
agent allows appraisal respect to refer to different aspects of human
character, such as Brogaard's reference to the extent a lover is
lovable. In the case of romantic love, this property we value would be
the ``[l]ovability'' of a person based on their
attributes.\footnote{Brogaard, \emph{Friendship Love and Romantic Love.}
  171.} Thus, romantic love is expressed when we love our beloved
``\emph{in their role as our romantic interest or partner,}'' and our
friends ``\emph{in their role as our friend}.''\footnote{Darwall, ``Two
  Kinds of Respect.'' 43.} This means there is not necessarily a set of
values against which we evaluate and determine whether to give love to
our lovers. Instead, we appraise our lovers by evaluating their ability
to demonstrate the properties we value in them.

Individual people love romantically and authentically when they find
those fulfilling the role of a romantic partner lovable in that role.
Their character must be that of a romantically lovable person and the
character of a lovable romantic partner that is constituted by
``dispositions to act for certain reasons [\ldots] to act, and in
acting to have certain reasons for acting.''\footnote{Darwall, ``Two
  Kinds of Respect.'' 43.} A lover's reasons for being lovable are just
as important as their lovability. Baked into Brogaard's account is the
idea that one cannot feign being ``lovable'' to secure things other than
loving their partner and being the best romantic partner possible.

This clearly picks out the issue of the pervasive love story's lack of
authenticity discussed earlier. Those engaging in heterosexual love
simply have too many inauthentic reasons for pursuing love in the first
place to be said to be prima facie able to love in a way that
demonstrates and is constituted by the right kind of respect for their
partner. This is also significant in bolstering my later argument
describing why the artificial love story mandated by patriarchy's system
of compulsory heterosexuality causes most men and women to have
inauthentic reasons for wanting to engage in love. ``Love'' as it is now
understood only facilitates and necessitates one's trying to be
\emph{perceived as} a lovable partner as opposed to their pursuit of
\emph{actually being} a lovable partner.

Brogaard then clarifies that that which determines one's lovability in
the role of a romantic partner is based on cultural and individual
scripts.\footnote{Brogaard, \emph{Friendship Love and Romantic Love}.
  172.} These scripts refer to:

\begin{quote}
structures comprising social roles, common knowledge, and norms and
guidelines that shape our perception, thinking, and action and guide our
interaction with others\ldots.Whereas cultural scripts are
\emph{constructs of the culture in which we are embedded}, individual
scripts are products of individual socialization, which includes our
\emph{upbringing and personal experiences}. [Emphasis added]
\end{quote}

\noindent One of these cultural scripts can thus be undeniably said to be Rich's
compulsory heterosexuality as it utterly determines, defines, and
enforces a specific kind of love that individuals and communities alike
struggle to free themselves from. As made evident by Rich's explanation
of the power and depth of compulsory heterosexuality, in terms of
heterosexism it seems the line between cultural and individual scripts
is quite blurred. If one were raised in a society that only ever talks
about the delight of cheese and never mentions broccoli except in a
disapproving manner, it is likely that would contribute to one's marked
(coerced) ``preference'' for cheese and unthinking hatred of broccoli.
It is in a manner similar to this that people are coerced into only
considering heterosexual love as a viable love, and thus it cheapens any
heterosexual love projects in which they attempt to engage.

Brogaard goes on to compare the impact of patriarchy and matriarchy on
concepts of shame, romantic love, and friendship love. While not the
direction in which she takes her argument, Brogaard thus provides a
theory of love that helps elucidate the inability of women and men to
truly love each other under patriarchy as the world stands by basing her
theory on appraisal respect. In Section IV, I will show how this also
gives us a roadmap with which to seek healthier, more authentic
relationships.

\section{Men and Women Cannot Love Each Other\ldots}

The cultural scripts of patriarchy and compulsory heterosexuality thus
make it so that men and women cannot authentically love each other.
Shulamith Firestone argues women must love ``not only for healthy
reasons but actually to validate their existence.''\footnote{Firestone,
  \emph{The Dialectic of Sex: The Case for Feminist Revolution}. 155}
Rich clearly thinks compulsory heterosexuality relegates women to that
same fate of engaging in heterosexual love not for authentic or healthy
reasons, but because women have to come to ``perceive ourselves as
sexual prey'' and grow into ``male identification.''\footnote{Rich,
  ``Compulsory Heterosexuality and Lesbian Existence.'' 642.} This
elucidates the fact that women are not held as ends-in-themselves and
cannot \emph{be} without first being defined by men. The romantic
pursuit of men on the part of women is then not genuine, but necessarily
motivated and calculated so as to ensure a connection to any kind of
subjectivity. This kind of motive, to no fault of the woman's own,
negates any authenticity her love could hold for a man. The influence of
patriarchy in negating her subjectivity and the influence of compulsory
heterosexuality in negating her choice to explore other forms of
romantic love negate her ability to consider men as possibly lovable in
the role of lover, and thus her ability to love men.

Conversely, there is no way for a man to gauge the actual lovability of
a woman because men need to fall in love with ``\emph{more} than
woman.''\footnote{Firestone, \emph{The Dialectic of Sex: The Case for
  Feminist Revolution}. 255} They must engage in a hyper-idealization of
women so as to be able to justify their loving someone who they are
taught can only serve to siphon their societal power and offer minimal
social status in return. Brogaard's account being one characterized by a
goal-oriented emotion similarly recognizes that idealization is at play
because to love is to desire to engage in love with the beloved `\,``or,
in any case, some idealized version of her or him.''\,'\footnote{Brogaard,
  \emph{Friendship Love and Romantic Love}. 165} Women then become
homosocial status symbols for men to prove to other men they are correct
and healthy in their ability to fulfill their role as a heterosexual man
in society.

Similar to women, men cannot consider other sexualities and are chained
to women. Jane Ward's terminology of the ``misogyny paradox'' describing
``men's simultaneous desire for and hatred of women'' dictated and
demanded by compulsory sexuality illustrates this well.\footnote{Ward,
  \emph{The Tragedy of Heterosexuality,} 33} Desire for women is thus
expected and forced out of men while women are presented as people
unworthy of respect in and of themselves. This makes evident that if
someone's lovability is based on the appraisal of their performance in
their role as a lover, it is impossible for men to see women as lovable
in romantic roles because their own participation in love is more a
fulfillment of duty than an interest in the person.

We know that femininity and the gathering of women together pose a
threat to patriarchy as a site of consciousness-raising. Men are
encouraged to distrust and destroy femininity because they are told it
is not ``manly'' and that it would mean the end of their supremacy.
Thus, men cannot love women because they cannot view them as those
capable of being lovable as romantic interests but instead objects meant
to be defined by men. Since women are taught to see men as that which
defines them and not those capable of being lovable as romantic
interests, women cannot be said to be able to love men either.

Objectifying women is key in affirming women's subjugation because men's
``identification with women (and what it means to be female) helps
remove the symbolic distance that enables men to depersonalize the
oppression of women.''\footnote{Bird, ``Welcome to the Men's Club:
  Homosociality and the Maintenance of

  Hegemonic Masculinity.'' 123.} In the same way that exploring the
lesbian continuum might grant women subjectivity, if men identified too
much with women and their own femininity, patriarchy would be disrupted
because men would begin to see women as subjects. Patriarchy instead
relies on a feedback loop of men necessarily objectifying women to
affirm women's subjugation, and women being subjugated because they are
objectified.

To love someone ``\emph{in their role as our romantic interest or
partner}'' would necessitate that the consideration of this type of role
for men or women were ever offered.\footnote{Brogaard, \emph{Friendship
  Love and Romantic Love}. 172} Men are instead effectively given the
roles of protector, abuser, or person meant to be appeased by women
according to patriarchy's love story. Compulsory heterosexuality takes
no interest in actually determining that men be viable love interests
for women, but instead that they be the only, inescapable
option\footnote{The usage of the word ``option'' is itself dubious in
  that it implies there is a choice between several options, whereas in
  compulsory heterosexuality, clearly the only model of romantic
  ``love'' allowed is the commitment of a man to a woman.} available.

The lack of choice and over exaggeration of a woman's lovable
characteristics so as to justify losing power cannot be said to
constitute love for a woman on a man's part. The lack of choice and lack
of an expectation for men to be lovable romantic interests to women
cannot be said to constitute love for a man on a woman's part either.

If Brogaard is correct that love is an emotion based on one's ability to
see their partner as lovable, or someone deserving of love, then it
seems men and women cannot yet love each other. There is no appraisal
respect between men and women as compulsory heterosexuality does not
allow it. In being told that women and men \emph{ought} to love each
other, women cannot see men as romantic partners or vice versa, and they
ultimately \emph{cannot} love each other.

\subsection{Can Men and Women Be Friends?}

This influences our cultural scripts surrounding friendship love as
well. Friendship love is impacted by compulsory heterosexuality because
finding a friend of the opposite sex authentically/genuinely ``lovable''
in their role as a friend is not allowed under patriarchy. It is
required that men and women expect to be engaged in claimant, not loving
or friendly, relationships with each other. Since the dominant cultural
scripts dictate that friendship is non-sexual and since Brogaard and I
want to say that one should value a friend in their role as a friend,
heterosexual friendships go unconsidered by patriarchy as a possibility.
Stories portrayed in social and traditional media rarely (if ever)
depict friendships between men and women that have no romantic or sexual
connotations, but that do have a friendship intimacy. Friendship
intimacy with those of one's own gender is already discouraged, but
authentic friendship between genders is such an unconsidered project
that it simply does not appear. The inability to regard each other with
appraisal respect also negates men and women's ability to define each
other as lovable friend interests.

It is important men and women find a way to love each other as friends
because that would be another key step in making authentic romantic love
possible. It would reject the implied tenet of romantic love that says
it must be sexual, and that anything else is simply friendship. All of
these forces heavily limit who and how we love, and if one of these
forces can be rejected in the hopes of securing a better, more authentic
love; then it seems all of them can be rejected. In fact, all of them
\emph{must} be eradicated before we can love. Men and women cannot
authentically love each other as romantic partners or friends.

\section{\ldots Yet. What We Ought to do to be Able to Love.}

So, there are forces that make it impossible for the majority of
heterosexual love projects to be called authentic love. These forces
include compulsory heterosexuality and the lack of freedom it allows in
choosing\footnote{Some have questioned what this focus on choice might
  mean for arranged marriages. I am not at all arguing that authentic
  romantic love cannot grow out of such environments (if the other
  oppressive constraints I discuss were to be properly dismantled)
  because there is a choice still at work behind love in such
  situations. One could have an arranged marriage to another and never
  love them or choose to love them, meaning one could also choose to
  love them.} partners, patriarchy actually rewarding those who do not
hold appraisal respect for their lovers, and the harmful representations
of love as something necessarily difficult.

\subsection{Navigating and Transgressing Against Compulsory
Heterosexuality; the Lesbian Continuum}

Rich offers a method to solve the first of these issues, namely, the
lesbian continuum. The lesbian continuum directly transgresses against
compulsory heterosexuality and patriarchy by encouraging female
friendships and sensual relationships between women. The basic idea is
that women can actually seek love from men if they love other members of
their gender and themselves enough to foster a sort of subjectivity and
appraisal respect for themselves as lovable to engage in romantic
projects with those of the opposite sex. It also encourages the
``bonding against male tyranny, the giving and receiving of practical
and political support; [and]\ldots\emph{marriage
resistance}.''\footnote{Rich, ``Compulsory Heterosexuality and Lesbian
  Existence.'' 648.} These are all actions praised by various feminist
consciousness raising movements and resistance movements generally. It
is hard to change anything if one is not supported by others who are
oppressed in the same way they are, and it is hard to even recognize an
issue regarding a community in the first place if communication between
those in the community is so divided. This is why consciousness raising
efforts for any social justice movements are suppressed; there is power
in community.

The lesbian continuum suggests there should be a similar continuum for
men. Many cultures outside of the WASP (White Anglo-Saxon Protestant)
cultures of the U.S. and U.K. encourage physical and emotional intimacy
between men. This is largely not the case in the U.S. and the U.K., but
it is also not the case that increased homosocial male intimacy has seen
widespread acceptance of queer men in these societies. Men need to value
themselves and other men as people who can be evaluated in terms of
their lovability as well. This might look like individual men putting
value in their exploration of their femininity and their increased
emotional vulnerability with each other. These endeavours would likely
lessen their need to objectify women and would succeed in freeing them
to engage in love as per Hannah Arendt's declaration, ``If men wish to
be free, it is precisely sovereignty they must renounce.''

While the first step would be encouraging homosocial bonding between
women and homosocial bonding between men, this would not be enough to
introduce queer relationships as being just as viable as heterosexual
ones. It seems there would need to be ongoing efforts to ensure the
equal treatment of queer love projects as viable in affirming the
viability of their heterosexual counterparts. This will not only make
authentic heterosexual love possible, but also authentic queer love more
accessible. It is not clear that compulsory heterosexuality benefits
people, and instead only benefits bureaucratic bodies interested in
distracting. Outside of maintaining cultures of self-policing encouraged
by cruel conceptions of ``morality'', compulsory heterosexuality just
greatly cheapens all types of love projects. ``Love'' is then about
aligning ourselves with others as to ensure our capital. Ridding
ourselves of this oppressive force would make both queer and
heterosexual love projects more authentic because neither could be
construed as a reaction to a greater societal force, but instead an
expression of intimacy that looks upon our lovers with love and not
exploitation.

\subsection{Conflating Conflict and Sacrifice with Love}

Does this all mean that if you have a partner and you are engaged in a
heterosexual love project, you do not love them? No, not necessarily. If
you have invested properly in yourself and your intimate relationships
with those of various identities, you have hopefully taught yourself how
to love others for their lovability. This is much, much rarer than we
take it to be; and there are thus many love projects that lack
authenticity entirely. Since one can and must navigate within such
oppressive forces\footnote{Again, assuming we have some minimal amount
  of free will.}, and because we can think of examples in our lives of
authentically loving heterosexual projects in which both people clearly
love and respect each other as lovers, love can exist under such
constraints.

How we are taught to love is an extremely harmful shame. I have argued
that we must educate ourselves and properly invest in our homosocial
relationships so as to even be \emph{able} to love. I am not arguing
that romantic love is unnatural. The need to love and be loved is likely
innate for many, but how we are taught to construct and pursue it is
completely learned. All the expectations of monogamy, heterosexuality,
etc. are taught. The supposed goal of ``love'' is also taught. We are
told that the goal of love projects is overcoming strife regarding your
love project or loving your lover in some sense \emph{in spite} of who
they are and the role they play in your life. Part of this love in spite
of who the other is has to do with their gender identity in relation to
your own, as discussed. The other issue at work in this problematic love
story is the idea that authentic love should be difficult, or that
``true'' love comes about when one makes sacrifices for their lover. It
seems true that one needs to be \emph{willing} to sacrifice and suffer
for their loved one to be said to love them, but for that to be a
necessary part of the love or that which proves the love is inauthentic
and unhealthy.

I agree with Brogaard that authentic love should come in one's ability
to evaluate their lover in their role as a lover. Unfortunately, we are
taught that ``love'' is something we must struggle to achieve, and that
big shows of passion and extremely costly and impractical gestures are
the most romantic. These things can be effective displays of affection,
and because I also agree with Brogaard that love is goal-oriented, it
makes sense that maintaining and expressing love necessitates some form
of extra effort at least occasionally. However, that being the
\emph{only} and most \emph{widely accepted} way of demonstrating one's
true love makes the goal of love projects deeply problematic. Love
becomes pure performance, a Romeo and Juliet feat of tragic
experience.\footnote{Of course, many agree that this story ultimately
  depicts an unnecessary and unfortunate amount of self-sacrifice.
  However, since many cultures have stories whose structure and outcome
  is similar to theirs, I take it to be a good indicator of the fact
  that there is a common belief in true love necessarily being hard-won
  is true.} If you respected your lover for their lovability and as
subjects worth respect generally, should you want to make them suffer?
Surely not. Similarly, they should not want you to suffer, and you
should not want them to want you to suffer for them. This need to prove
your love comes from a learned insecurity, not only on an interpersonal
level, but a societal one as well.

Authentic love can come from certain relationships in which there is
some kind of power asymmetry between the partners, or some difficult
force they must overcome. ``Loving'' someone \emph{because} you enjoy
your one-sided power over them or \emph{because} you enjoy their
one-sided power over you seems like pursuing the wrong kind of goal in
your love project. Subordination and domination might be aspects of
organizing all kinds of relationships, but authentic love cannot have
that as its core goal because that is not loving someone with the proper
respect for them as lovable people. How subordination and domination
configure into sex might be a separate matter, depending on how closely
connected one understands sex and love to be. This is an interesting
topic, but out of the scope of this paper.

There is also the matter of comparison of one's partner and love project
to those of another. This seems to kill love. Envy of this strain is not
an issue specific to romantic love, though, and it is unclear as a
result that we can relate to others without \emph{any} sense of
comparison \emph{ever}. All of the societal forces described encourage
competition and a sense of there being ``losers'' and ``winners'' in
romantic love, which is problematic in all of love's forms. Presumably
this could be alleviated at least somewhat by learning to respect
oneself and others and dismantling the ``love as conflict'' story. Envy
of this kind might be possible to completely disentangle from our
connections to others, but I am unsure. That might require the type of
deep introspection that reveals to one that no connections are necessary
or worthwhile at all.

Authentic romantic love as a standalone project should have loving your
partner in their role as a lover as its goal. No societal force under
which we engage in romantic love supports or allows for this, so it is
nearly impossible to love authentically. However, authentic heterosexual
love is possible if one undertakes the labor intensive but crucial,
intentional unlearning of the oppressive stories we are told and the
intentional reteaching of how to actually love each other.

\section{Conclusion}

Men and women cannot be said to love each other romantically nor as
friends under compulsory heterosexuality, but that does not mean it is
essentially impossible, just impossible under current societal
conditions. This is because men and women cannot idealize each other in
such a way that they can actually evaluate the other's lovability as
romantic partners or friends. Solidarity of any kind is threatening to
oppressive social structures, but if men and women want to love each
other authentically as friends and lovers, solidarity is key. First,
individual men and women must invest in their respect for themselves and
their homosocial relationships. Then, they can evaluate each other in
their roles as lovable lovers, and lovable friends.

\refsection

\begin{hangparas}{\hangingindent}{1}
Arendt, Hannah. ``What is Freedom?'' in \emph{Between Past and Future},
New York, Penguin Books, 1992 [1977].

Bauer, Nancy. \emph{Simone de Beauvoir, Philosophy, \& Feminism}. New
York: Columbia University Press, 2001.

Bird, Sharon R. ``Welcome to the Men's Club: Homosociality and the
Maintenance of Hegemonic Masculinity.'' \emph{Gender \&amp; Society}, vol. 10, no. 2, Apr. 1996, pp. 120--132,
\newline
\url{https://doi.org/10.1177/089124396010002002.}

Brogaard, Berit (2022). \emph{Friendship Love and Romantic Love}. In
Diane Jeske (ed.), The Routledge Handbook of Philosophy of Friendship. New York, NY: Routledge. pp. 166-178.

Darwall, Stephen L. ``Two Kinds of Respect.'' \emph{Ethics}, vol. 88,
no. 1, Oct. 1977, pp. 36--49,
\newline
\url{https://doi.org/10.1086/292054.}

Firestone, Shulamith. \emph{The Dialectic of Sex: The Case for Feminist
Revolution}. William Morrow and Company, 1971.

Ward, Jane. \emph{The Tragedy of Heterosexuality}. New York University
Press, 2020.

Weeks, Kathi. \emph{The Problem with Work: Feminism, Marxism, Antiwork
Politics, and Postwork Imaginaries}. Duke University Press, 2011.

Rich, Adrienne. ``Compulsory Heterosexuality and Lesbian Existence.''
\emph{Signs}, vol. 5, no. 4, 1980, pp. 631--60. \emph{JSTOR}, accessed 21 Sept. 2023,
\newline
\url{http://www.jstor.org/stable/3173834.} 

Srinivasan, Amia. \emph{The Right to Sex: Feminism in the Twenty-First
Century}. Picador/Farrar, Straus and Giroux, 2022.
\end{hangparas}
